\documentclass{resume} % Use the custom resume.cls style
\setlength{\parskip}{6pt}%

\newcommand{\itemsepval}{-6pt}

\usepackage[left=0.5in,top=0.5in,right=0.5in,bottom=0.5in]{geometry} % Document margins
\newcommand{\tab}[1]{\hspace{.2667\textwidth}\rlap{#1}} 
\newcommand{\itab}[1]{\hspace{0em}\rlap{#1}}
\usepackage{etoolbox}
\usepackage{minted}
% \AtBeginEnvironment{itemize}{\setlength\parskip{6pt}}
\AtBeginEnvironment{itemize}{\vspace*{-\dimexpr\parskip\relax}}
% \AfterEndEnvironment{itemize}{\vspace*{-\dimexpr\parskip\relax}}

\name{Zhou, Qi} 

\address{+44 07710 799432 \\ +86 185 5254 1785}
% \address{\url{+44 07710 799432} \\ \url{+86 185 5254 1785}} 
\address{\url{zhouqi008012@gmail.com} \\ \url{Q.Zhou-24@sms.ed.ac.uk} } 


\begin{document}


\begin{rSection}{Education}

{\bf  University of Edinburgh}, Computer Science (BSc Hons)\hfill {2019 - 2023}

\end{rSection}

\begin{rSection}{Work Experience}

    \textbf{Asteria} \hfill Sept 2020 - Feb 2021\\
    \url{https://www.asteria-space.com/} \hfill \par
    \begin{itemize}
        \itemsep \itemsepval {} 
        \item[-] Came up with possible solution for attaching a camera on a single-board computer which will be launched into space.
    \end{itemize}
     
    \textbf{Codeplay} \hfill June 2022 - Sept 2022 \\
    \url{https://www.codeplay.com/} \hfill \par
    \begin{itemize}
        \itemsep \itemsepval {} 
        \item[-] Worked on ComputeArota (CA), a toolkit implementing cross-platform OpenCL, Vulkan and etc.
        \item[-] Added RV32 and half float (Zfh extension) support to the simulator in CA, such that RV32 or Float16 instructions can be generated by CA and simulated using SPIKE. 
        % \item[-] Discovered and patched some test failing due to float point precision problems. 
        \item[-] Learnt about Sollya’s fpminimax function that computes a polynomial approximation for float point operations, then investigated some test failures caused by float point precision issues.
        \item[-] Implemented a faster and more intuitive replacement of SPIKE for CA using RISCV-64 QEMU with a Linux operating system, where the host (client) communicates with simulator (server) by sockets. 
    \end{itemize}
      
    \textbf{Teaching assistant} \hfill Sept 2022 - May 2023     \\
    Demonstrator in workshops and labs for these courses: \hfill \par
    \begin{itemize}
        \itemsep \itemsepval {} 
        \item[-] Computer Architecture and Design (INFR10076), 
        \item[-] Compiling Techniques (INFR10065).
    \end{itemize}


\end{rSection} 


\begin{rSection}{Projects}

    \textbf{Lorenz Attractor} \hfill \\
    \url{https://github.com/AOIDUO/LorenzAttractor/} \hfill \par
    \begin{itemize}
        \itemsep \itemsepval {} 
        \item[-] A animated figure of Lorenz attractor written in Haskell.
        \item[-] Demo: \url{https://homepages.inf.ed.ac.uk/wadler/fp-competition-2019/#(16)}.
    \end{itemize}

    \textbf{RISC-V Processor} \par
    \begin{itemize}
        \itemsep \itemsepval {} 
        \item[-] Programs a Xilinx PYNQ demo board into a working RV32IM processor.
        \item[-] Classic 5-stage pipeline and speculative result forwarding.
        \item[-] Fast 32-bit multiplier by utilizing FPGA's builtin 16-bit multiplier in parallel.
    \end{itemize}

    \textbf{Turing Machine Emulator} \hfill \\ 
    \url{https://github.com/AOIDUO/RegisterMachineEmulator/} \hfill \par
    \begin{itemize}
        \itemsep \itemsepval {} 
        
        \item[-] A emulator for the register variant of turing machine with only 2 instructions: \mintinline{cpp}{inc} and \mintinline{cpp}{decjz}.
        \item[-] Has a parser that recognizes instruction source code in BNF style.
        \item[-] Planning to complete the support of macro in the future.
    \end{itemize}

    \textbf{SIMD Support for LLVM MLIR Presburger library - In progress} \hfill \\
    \url{llvm/llvm-project/blob/main/mlir/lib/Analysis/Presburger/Simplex.cpp} \hfill \par
    \begin{itemize}
        \itemsep \itemsepval {} 
        \item[-] This library performs overflow-checked multiplication and addition on small and sparse matrix. 
        \item[-] Compute 52-bit integer using FPU could be fast, because: 
        \begin{itemize}
            \itemsep \itemsepval {} 
            \item[-] Fraction part of double precision float point number is exactly 52 bits,
            \item[-] Exploits fused-multiply-add,
            \item[-] Float point overflow and imprecision checking is convenient.
        \end{itemize}
    

    \end{itemize}
\end{rSection} 

\begin{rSection}{Skills}
    - Have experience with GNU/Linux, \\
    - Familiar with Java, Agda, Haskell, Python, Shell, Verilog and C++, \\
    - Capable of building embedded widgets with single board computers and PCDIY.

\end{rSection} 


\end{document}
