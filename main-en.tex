\documentclass{resume} % Use the custom resume.cls style
\setlength{\parskip}{6pt}%

\newcommand{\itemsepval}{-6pt}
\newcommand{\code}{\mintinline{html}}

\usepackage[left=1in,top=1in,right=1in,bottom=1in]{geometry} % Document margins
\newcommand{\tab}[1]{\hspace{.2667\textwidth}\rlap{#1}} 
\newcommand{\itab}[1]{\hspace{0em}\rlap{#1}}
\usepackage{etoolbox}
\usepackage{minted}
% \AtBeginEnvironment{itemize}{\setlength\parskip{6pt}}
\AtBeginEnvironment{itemize}{\vspace*{-\dimexpr\parskip\relax}}
% \AfterEndEnvironment{itemize}{\vspace*{-\dimexpr\parskip\relax}}

\name{Zhou, Qi} 

\address{+44 07710 799432 \\ +86 185 5254 1785}
% \address{\url{+44 07710 799432} \\ \url{+86 185 5254 1785}} 
\address{\url{zhouqi008012@gmail.com} \\ \url{Q.Zhou-24@sms.ed.ac.uk} } 


\begin{document}


\begin{rSection}{Education}

{\bf  University of Edinburgh}, Computer Science (BSc Hons)\hfill {2019 - 2023} \\
Average grade 74 (A3) (equivalent GPA 4.0)

\end{rSection}

\begin{rSection}{Work Experience}


     
    \textbf{Codeplay} \hfill June 2022 - Sept 2022 \\
    \url{https://www.codeplay.com/} \hfill \par
    \begin{itemize}
        \itemsep \itemsepval {} 
        \item[-] Worked on \code{ComputeArota} (\code{CA}), a toolkit implementing heterogeneous cross-platform computing.
        %  \code{OpenCL}, \code{Vulkan} and etc.
        \item[-] Added \code{RV32} and \code{Zfh} extension (half float) support to the simulator in \code{CA}, such that \code{RV32} or \code{Float16} instructions can be generated by \code{CA} and simulated using \code{SPIKE}.
        % \item[-] Discovered and patched some test failing due to float point precision problems. 
        \item[-] Learnt about \code{Sollya }’s \code{fpminimax} function that computes a polynomial approximation for floating point operations, then investigated some test failures caused by floating point precision issues.
        \item[-] Implemented a faster and more intuitive replacement of \code{SPIKE} for \code{CA} using \code{RISCV-64 QEMU} with a Linux operating system, where the host (client) communicates with simulator (server) using sockets. 
    \end{itemize}
    
    \textbf{Teaching assistant} \hfill Sept 2022 - May 2023     \\
    For these courses I achieved high grade, therefore being qualified for teaching support roles:
    
    \begin{itemize}
        \itemsep \itemsepval {} 
        \item[-] \textbf{Computer Architecture and Design} (INFR10076)
        \begin{itemize}
            \itemsep \itemsepval {} 
            \item[-] Demonstrator in workshops to help students with the 5-stage pipeline \code{RV32IM} core coursework: 
            \begin{itemize}
                \itemsep \itemsepval {} 
                \item[1.] Get familiar with Xilinx PYNQ FPGA board through toy examples. 
                \item[2.] Implement ALU and register file according to \code{RV32IM} specification. 
                \item[3.] Integrate speculative result forwarding into the pipeline.
                \item[4.] Analysis critical path and optimize accordingly, for example, execute fast \code{int32} multiplication by utilizing FPGA's builtin \code{int16} multiplier in parallel.        
            \end{itemize}
    
        \end{itemize}

        \item[-] \textbf{Compiling Techniques} (INFR10065)
        \begin{itemize}
            \itemsep \itemsepval {} 
            \item[-] The coursework is a compiler from \code{Python} to \code{RISC-V} assembly using a ``MLIR-lite'' framework: \code{xDSL} \url{https://xdsl.dev/}.
            \item[-] Contributed in coursework template design and bug fixes. 
            \item[-] Answer questions for students during workshops.
        \end{itemize}
    \end{itemize}


    \textbf{Asteria} \hfill Sept 2020 - Feb 2021\\
    \url{https://www.asteria-space.com/} \hfill \par
    \begin{itemize}
        \itemsep \itemsepval {} 
        \item[-] Came up with possible solution for attaching a camera on a single-board computer which will be launched into space.
    \end{itemize}

\end{rSection} 


\begin{rSection}{Projects}
    \textbf{SIMD Support for LLVM MLIR Presburger library - In progress} \hfill \\
    \url{llvm/llvm-project/blob/main/mlir/lib/Analysis/Presburger/Simplex.cpp} \hfill \par
    \begin{itemize}
        \itemsep \itemsepval {} 
        \item[-] This library performs overflow-checked multiplication and addition on small matrix. 
        \item[-] Compute \code{23} bits or \code{52} bits integer using FPU could be fast, because: 
        \begin{itemize}
            \itemsep \itemsepval {} 
            \item[-] Mantissa part of single and double precision floating point number is exactly \code{23} and \code{52} bits,
            \item[-] Exploits fused-multiply-add,
            \item[-] Sufficient for small data, not likely to trigger \code{SIGFPE} and redirect to slow \code{BigInt} algorithms, 
            \item[-] Floating point overflow and imprecision checking is convenient.
        \end{itemize}
    \item[-] Discovered bugs in \code{llvm-mca} when it analysis \code{FMA} throughput on \code{zen3}. 
    \item[-] Analyzed performance characteristics using \code{google/benchmark} and \code{perf}.
    \end{itemize}

    % \textbf{RISC-V Processor} \par
    % \begin{itemize}
    %     \itemsep \itemsepval {} 
    %     \item[-] Programs a Xilinx PYNQ demo board into a working \code{RV32IM} processor.
    %     \item[-] Classic 5-stage pipeline and speculative result forwarding.
    %     \item[-] Fast \code{int32} multiplication by utilizing FPGA's builtin \code{int16} multiplier in parallel.
    % \end{itemize}
    
    \textbf{Ukulele Tuner} 
        \begin{itemize}
        \itemsep \itemsepval {} 
        \item[-] A coursework where a group of 9 work together on an assistive robotics project.
        \item[-] Best Commercialization Award!
        \item[-] Fully working handheld device, consists of a 3.5 inch touch screen, a microphone, a stepping motor and a raspberry pi, then connected to each other with a custom PCB and assembled in a 3D printed chassis. 
        \item[-] The process involves converting a string pluck recording into its corresponding vibrational frequency using Fast Fourier Transform. The resulting frequency is then compared to the expected frequency and motor rotates according to a lookup table. 
    \end{itemize}

    \textbf{Lorenz Attractor} \hfill \\
    \url{https://github.com/AOIDUO/LorenzAttractor/} \hfill \par
    \begin{itemize}
        \itemsep \itemsepval {} 
        \item[-] A animated figure of Lorenz attractor written in Haskell.
        \item[-] Demo: \url{https://homepages.inf.ed.ac.uk/wadler/fp-competition-2019/#(16)}.
    \end{itemize}

    \textbf{Turing Machine Emulator} \hfill \\ 
    \url{https://github.com/AOIDUO/RegisterMachineEmulator/} \hfill \par
    \begin{itemize}
        \itemsep \itemsepval {} 
        \item[-] A turing machine emulator written in \code{Python}.
        \item[-] Partially supports marco. 
        \item[-] Utilized parser provided by \code{xDSL} to recognize instruction source code in BNF style.
    \end{itemize}


\end{rSection} 

\begin{rSection}{Skills}
    - Practiced with \code{Java}, \code{Agda}, \code{Haskell}, \code{Python}, \code{Shell}, \code{Verilog} and \code{C++}, \\
    - Also experienced in \code{GNU/Linux}, \code{git}, \code{GDB}, CI, many other toolchains and software testing toolkits, \\
    % - Capable of building embedded widgets with single board computers and PCDIY.
    - Capable of collaborative work on big repository, \\
    - Learnt \emph{Programming Language Foundations in Agda}, familiar with formal verification and type theory. 
\end{rSection} 


\end{document}
